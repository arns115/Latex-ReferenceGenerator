\documentclass{article}
\usepackage{listings}
\usepackage{xcolor}
\usepackage{multicol}
\usepackage[a4paper, total={6.5in, 10in}]{geometry}
%New colors defined below
\definecolor{codegreen}{rgb}{0,0,0}
\definecolor{codegray}{rgb}{0,0,0}
\definecolor{codepurple}{rgb}{0,0,0}
\definecolor{backcolour}{rgb}{1,1,1}

%Code listing style named "mystyle"
\lstdefinestyle{mystyle}{
  backgroundcolor=\color{backcolour},   commentstyle=\color{codegreen},
  keywordstyle=\color{codegreen},
  numberstyle=\tiny\color{codegray},
  stringstyle=\color{codepurple},
  basicstyle=\ttfamily\footnotesize,
  breakatwhitespace=false,         
  breaklines=true,                 
  captionpos=b,                    
  keepspaces=true,                 
  numbers=left,                    
  numbersep=5pt,                  
  showspaces=false,                
  showstringspaces=false,
  showtabs=false,                  
  tabsize=2
}

%"mystyle" code listing set
\lstset{style=mystyle}

\title{Cheatsheet}
\date{2023}
\author{ICPC}
\begin{document}

\maketitle
\begin{multicols}{2}\section{Graph}
\subsection{bfs}
\lstset {    language=C++,
    basicstyle=\small\ttfamily,
    numbers=left,
    breaklines=true,
    tabsize=4}
\begin{lstlisting}[linewidth=\columnwidth,breaklines=true,language=C++]
int n, m;
vector<int> adj[MAX_N], dist;

void bfs(int s) {
    dist.assign(n + 1, -1);
    queue<int> q;
    dist[s] = 0; q.push(s);
    while (q.size()) {
        int u = q.front(); q.pop();
        for (int v : adj[u]) {
            if (dist[v] == -1) {
                dist[v] = dist[u] + 1;
                q.push(v);
            }
        }
    }
}
\end{lstlisting}
\subsection{dfs}
\lstset {    language=C++,
    basicstyle=\small\ttfamily,
    numbers=left,
    breaklines=true,
    tabsize=4}
\begin{lstlisting}[linewidth=\columnwidth,breaklines=true,language=C++]
const int MAX_N = 1e5 + 5;
const ll MOD = 1e9 + 7;
const ll INF = 1e9;

int n, m, vis[MAX_N];
vector<int> adj[MAX_N];

void dfs(int u) {
    vis[u] = 1;
    for (int v : adj[u]) {
        if (vis[v]) continue;
        dfs(v);
    }
}
\end{lstlisting}
\subsection{flood fill}
\lstset {    language=C++,
    basicstyle=\small\ttfamily,
    numbers=left,
    breaklines=true,
    tabsize=4}
\begin{lstlisting}[linewidth=\columnwidth,breaklines=true,language=C++]
int n, m, visited[MAX_N][MAX_N];
char grid[MAX_N][MAX_N];
 
const int di[] = {1, 0, -1, 0};
const int dj[] = {0, -1, 0, 1};
 
bool valid(int i, int j) {
    return i >= 0 && j >= 0 && i < n && j < m && grid[i][j] == '.';
}
 
void dfs(int i, int j) {
    visited[i][j] = 1;
    for (int k = 0; k < 4; k++) {
        int ni = i + di[k], nj = j + dj[k];
        if (valid(ni, nj) && !visited[ni][nj]) {
            dfs(ni, nj);
        }
    }
}
\end{lstlisting}
\subsection{Kruskal}
\lstset {    language=C++,
    basicstyle=\small\ttfamily,
    numbers=left,
    breaklines=true,
    tabsize=4}
\begin{lstlisting}[linewidth=\columnwidth,breaklines=true,language=C++]
int n, m, par[MAX_N];
vector<array<int,3>> edges;
 
int find(int u) {
    return u == par[u] ? u : par[u] = find(par[u]);
}
 
void unite(int u, int v) {
    par[find(u)] = find(v);
}

void kruskal() {
    sort(edges.begin(), edges.end());
    for (int i = 1; i <= n; i++) par[i] = i;

    ll cnt = 0, cost = 0;
    for (auto [w, u, v] : edges) {
        u = find(u), v = find(v);
        if (u != v) {
            par[u] = v;
            cost += w;
            cnt++;
        }
    }
    if (cnt == n - 1) cout << cost << "\n";
    else cout << "IMPOSSIBLE\n";
}

\end{lstlisting}
\subsection{Dijkstra}
\lstset {    language=C++,
    basicstyle=\small\ttfamily,
    numbers=left,
    breaklines=true,
    tabsize=4}
\begin{lstlisting}[linewidth=\columnwidth,breaklines=true,language=C++]
int n, m;
vector<ar<int,2>> adj[MAX_N];
vector<ll> dist;

void dijkstra(int s) {
    dist.assign(n + 1, LINF);
    priority_queue<ar<ll,2>, vector<ar<ll,2>>, greater<ar<ll,2>>> pq;
    dist[s] = 0; pq.push({0, s});
    while (pq.size()) {
        auto [d, u] = pq.top(); pq.pop();
        if (d > dist[u]) continue;
        for (auto [v, w] : adj[u]) {
            if (dist[v] > dist[u] + w) {
                dist[v] = dist[u] + w;
                pq.push({dist[v], v});
            }
        }
    } 
}

\end{lstlisting}
\subsection{Bellman-Ford}
\lstset {    language=C++,
    basicstyle=\small\ttfamily,
    numbers=left,
    breaklines=true,
    tabsize=4}
\begin{lstlisting}[linewidth=\columnwidth,breaklines=true,language=C++]
int n, m, par[MAX_N];
vector<ar<ll,2>> adj[MAX_N];
vector<ll> dist;

void bellman_ford(int s) {
    dist.assign(n + 1, LINF);
    dist[s] = 0;
    for (int i = 0; i < n - 1; i++) {
        for (int u = 1; u <= n; u++) {
            for (auto [v, w] : adj[u]) {
                if (dist[u] + w < dist[v]) {
                    par[v] = u;
                    dist[v] = dist[u] + w;
                }
            }
        }
    }
}

void cycle_detect() {
    int cycle = 0;
    for (int u = 1; u <= n; u++) {
        for (auto [v, w] : adj[u]) {
            if (dist[u] + w < dist[v]) {
                cycle = v; 
                break;
            }
        }
    }
    if (!cycle) cout << "NO\n";
    else {
        cout << "YES\n";
        // backtrack to print the cycle
        for (int i = 0; i < n; i++) cycle = par[cycle];
        vector<int> ans; ans.push_back(cycle);
        for (int i = par[cycle]; i != cycle; i = par[i]) ans.push_back(i); 
        ans.push_back(cycle);
        reverse(ans.begin(), ans.end());
        for (int x : ans) cout << x << " ";
        cout << "\n";
    }
}
\end{lstlisting}
\subsection{Floyd Warshall}
\lstset {    language=C++,
    basicstyle=\small\ttfamily,
    numbers=left,
    breaklines=true,
    tabsize=4}
\begin{lstlisting}[linewidth=\columnwidth,breaklines=true,language=C++]
int n, m, par[MAX_N];
vector<ar<ll,2>> adj[MAX_N];
vector<ll> dist;

void bellman_ford(int s) {
    dist.assign(n + 1, LINF);
    dist[s] = 0;
    for (int i = 0; i < n - 1; i++) {
        for (int u = 1; u <= n; u++) {
            for (auto [v, w] : adj[u]) {
                if (dist[u] + w < dist[v]) {
                    par[v] = u;
                    dist[v] = dist[u] + w;
                }
            }
        }
    }
}

void cycle_detect() {
    int cycle = 0;
    for (int u = 1; u <= n; u++) {
        for (auto [v, w] : adj[u]) {
            if (dist[u] + w < dist[v]) {
                cycle = v; 
                break;
            }
        }
    }
    if (!cycle) cout << "NO\n";
    else {
        cout << "YES\n";
        // backtrack to print the cycle
        for (int i = 0; i < n; i++) cycle = par[cycle];
        vector<int> ans; ans.push_back(cycle);
        for (int i = par[cycle]; i != cycle; i = par[i]) ans.push_back(i); 
        ans.push_back(cycle);
        reverse(ans.begin(), ans.end());
        for (int x : ans) cout << x << " ";
        cout << "\n";
    }
}
\end{lstlisting}
\subsection{Korasaju}
\lstset {    language=C++,
    basicstyle=\small\ttfamily,
    numbers=left,
    breaklines=true,
    tabsize=4}
\begin{lstlisting}[linewidth=\columnwidth,breaklines=true,language=C++]
int n, m, scc, visited[MAX_N];
vector<int> adj[2][MAX_N], comp[MAX_N], dfn;
// adj[0] is the original graph, adj[1] is the transposed graph

// t = 0 means 1st pass, t = 1 means 2nd pass
void dfs(int u, int t) {
    visited[u] = 1;
    if (t == 1) comp[scc].push_back(u);
    for (int v : adj[t][u]) {
        if (!visited[v]) {
            dfs(v, t);
        }
    }
    if (t == 0) dfn.push_back(u);
}

void kosaraju() {
    for (int i = 1; i <= n; i++) {
        if (!visited[i]) {
            dfs(i, 0);
        }
    }
    memset(visited, 0, sizeof visited);
    for (int i = n - 1; i >= 0; i--) {
        if (!visited[dfn[i]]) {
            scc++;
            dfs(dfn[i], 1);
        }
    }
}
\end{lstlisting}
\subsection{Tarjan}
\lstset {    language=C++,
    basicstyle=\small\ttfamily,
    numbers=left,
    breaklines=true,
    tabsize=4}
\begin{lstlisting}[linewidth=\columnwidth,breaklines=true,language=C++]
int n, m, scc, dfsCounter;
int dfs_num[MAX_N], dfs_low[MAX_N], visited[MAX_N];
stack<int> st;
vector<int> adj[MAX_N], comp[MAX_N];

// Modified DFS
void tarjan(int u) { 
    dfs_low[u] = dfs_num[u] = dfsCounter++;
    visited[u] = 1;
    st.push(u);
    for (int v : adj[u]) {
        if (dfs_num[v] == -1) tarjan(v); 
        if (visited[v]) dfs_low[u] = min(dfs_low[u], dfs_low[v]); 
    }

    if (dfs_low[u] == dfs_num[u]) {
        scc++;
        while (st.size()) {
            int v = st.top(); st.pop();
            visited[v] = 0;
            comp[scc].push_back(v);
            if (v == u) break;
        }
    }
}

void dfs(int u) {
    visited[u] = 1;
    for (int v : adj[u])
        if (!visited[v])
            dfs(v);
}
\end{lstlisting}
\section{Dynamic Programming}
\subsection{Longest Increasing Subsequence(LIS)}
\lstset {    language=C++,
    basicstyle=\small\ttfamily,
    numbers=left,
    breaklines=true,
    tabsize=4}
\begin{lstlisting}[linewidth=\columnwidth,breaklines=true,language=C++]
int n; cin >> n;
vector<int> dp;
for (int i = 0; i < n; i++) {
    int x; cin >> x;
    auto it = lower_bound(dp.begin(), dp.end(), x);
    if (it == dp.end()) dp.push_back(x);
    else *it = x;
}
cout << dp.size() << "\n";
}
\end{lstlisting}
\subsection{Longest Common Subsequence(LCS)}
\lstset {    language=C++,
    basicstyle=\small\ttfamily,
    numbers=left,
    breaklines=true,
    tabsize=4}
\begin{lstlisting}[linewidth=\columnwidth,breaklines=true,language=C++]
string x, y; cin >> x >> y;
int n = x.size(), m = y.size();
int dp[n + 1][m + 1] = {};
for (int i = 1; i <= n; i++) {
    for (int j = 1; j <= m; j++) {
        if (x[i - 1] == y[j - 1]) 
            dp[i][j] = max(dp[i][j], dp[i - 1][j - 1] + 1);
        else
            dp[i][j] = max({dp[i][j], dp[i - 1][j], dp[i][j - 1]});
    }
}
cout << dp[n][m];
\end{lstlisting}
\section{Data Structures}
\subsection{Sparse table}
\lstset {    language=C++,
    basicstyle=\small\ttfamily,
    numbers=left,
    breaklines=true,
    tabsize=4}
\begin{lstlisting}[linewidth=\columnwidth,breaklines=true,language=C++]
int n, q;
int arr[MAX_N], dp[MAX_N][MAX_L];
 
void build_sparse_table() {
    for (int i = 1; i <= n; i++)
        dp[i][0] = arr[i];
    for (int j = 1; j < MAX_L; j++)
        for (int i = 1; i + (1 << j) <= n + 1; i++)
            dp[i][j] = min(dp[i][j - 1], dp[i + (1 << (j - 1))][j - 1]);
            // dp[i][j] = dp[i][j - 1] ^ dp[i + (1 << (j - 1))][j - 1];
}

int lg(int x) {
    return 32 - __builtin_clz(x) - 1;
}

int min_query(int l, int r) { // O(1)
    int k = lg(r - l + 1);
    return min(dp[l][k], dp[r - (1 << k) + 1][k]);
}

int xor_query(int l, int r) { // O(logn)
    int k = r - l + 1, ans = 0;
    for (int i = 0; i < MAX_L; i++) {
        if (k & (1 << i)) {
            ans ^= dp[l][i];
            l += (1 << i);
        }
    }
    return ans;
}

\end{lstlisting}
\subsection{Fenwick Tree}
\lstset {    language=C++,
    basicstyle=\small\ttfamily,
    numbers=left,
    breaklines=true,
    tabsize=4}
\begin{lstlisting}[linewidth=\columnwidth,breaklines=true,language=C++]
#define LSOne(S) ((S) & (-S))

int n, q;
ll ft[MAX_N];
 
void update(int x, int v) {
    for (; x <= n; x += LSOne(x))
        ft[x] += v;
}
 
ll sum(int x) {
    ll res = 0;
    for (; x; x -= LSOne(x)) 
        res += ft[x];
    return res;
}
 
ll rsq(int l, int r) {
    return sum(r) - sum(l - 1);
}

\end{lstlisting}
\subsection{Segment Tree}
\lstset {    language=C++,
    basicstyle=\small\ttfamily,
    numbers=left,
    breaklines=true,
    tabsize=4}
\begin{lstlisting}[linewidth=\columnwidth,breaklines=true,language=C++]
int n, q, arr[MAX_N], st[4 * MAX_N];
 
void build(int node, int start, int end) {
    if (start == end) {
        st[node] = arr[start];
        return;
    }
    int mid = (start + end) / 2;
    build(2 * node, start, mid);
    build(2 * node + 1, mid + 1, end);
    st[node] = min(st[2 * node], st[2 * node + 1]);
}
 
void update(int node, int start, int end, int idx, int val) {
    if (start == end) {
        arr[idx] = val;
        st[node] = val;
        return;
    }
    int mid = (start + end) / 2;
    if (idx <= mid) update(2 * node, start, mid, idx, val);
    else update(2 * node + 1, mid + 1, end, idx, val);
    st[node] = min(st[2 * node], st[2 * node + 1]);
}

int query(int node, int start, int end, int l, int r) {
    if (start > r || end < l) return INF;
    if (l <= start && end <= r) return st[node];
    int mid = (start + end) / 2;
    return min(query(2 * node, start, mid, l, r), query(2 * node + 1, mid + 1, end, l, r)); 
}
\end{lstlisting}
\subsection{Disjoint Set Union}
\lstset {    language=C++,
    basicstyle=\small\ttfamily,
    numbers=left,
    breaklines=true,
    tabsize=4}
\begin{lstlisting}[linewidth=\columnwidth,breaklines=true,language=C++]
int n, par[MAX_N], sz[MAX_N], num_grp;
 
int find(int u) {
    return u == par[u] ? u : par[u] = find(par[u]);
}
 
void merge(int u, int v) {
    u = find(u), v = find(v);
    if (u == v) return;
    if (sz[u] > sz[v]) swap(u, v);
    par[u] = v;
    sz[v] += sz[u];
    num_grp--;
}
\end{lstlisting}
\section{Strings}
\subsection{KMP}
\lstset {    language=C++,
    basicstyle=\small\ttfamily,
    numbers=left,
    breaklines=true,
    tabsize=4}
\begin{lstlisting}[linewidth=\columnwidth,breaklines=true,language=C++]
vector<int> prefix_func(string s) {
    int n = s.size();
    vector<int> f(n); 
    for (int i = 1; i < n; i++) {
        int j = f[i - 1];
        while (j && s[i] != s[j]) j = f[j - 1];
        f[i] = j + (s[i] == s[j]);
    }
    return f;
}
 
int cnt_occ(string s, string t) {
    string ts = t + "#" + s;
    int n = t.size(), m = s.size(), nm = ts.size();
    auto f = prefix_func(ts);
    int res = 0;
    for (int i = n + 1; i < nm; i++) res += (f[i] == n);
    return res;
}
 
\end{lstlisting}
\subsection{Rabin-Karp}
\lstset {    language=C++,
    basicstyle=\small\ttfamily,
    numbers=left,
    breaklines=true,
    tabsize=4}
\begin{lstlisting}[linewidth=\columnwidth,breaklines=true,language=C++]
mt19937 rng(chrono::steady_clock::now().time_since_epoch().count());
ll rand(ll l, ll r) {return uniform_int_distribution(l, r)(rng);}
const int BASE = rand(300, 1e9);
const int NUM_MOD = 2;
const ll MODS[] = {(int)1e9 + 2277, (int)1e9 + 5277};
struct Hash {
    vector<ll> H[NUM_MOD], P[NUM_MOD];
    Hash(string s) {
        for (int i = 0; i < NUM_MOD; i++) {
            P[i].push_back(1);
            H[i].push_back(0);
        }
        for (char c : s) {
            for (int i = 0; i < NUM_MOD; i++) {
                P[i].push_back(P[i].back() * BASE % MODS[i]);
                H[i].push_back((H[i].back() * BASE + c) % MODS[i]);
            }
        }
    } 
    ar<ll,NUM_MOD> operator()(int l, int r) {
        ar<ll,NUM_MOD> res;
        for (int i = 0; i < NUM_MOD; i++) {
            res[i] = (H[i][r + 1] - H[i][l] * P[i][r + 1 - l]) % MODS[i];
            if (res[i] < 0) res[i] += MODS[i];
        }
        return res;
    }
};

int cnt_occ(string s, string t) {
    int n = s.size(), m = t.size();
    Hash hs(s), ht(t);
    int res = 0;
    for (int i = 0, j = m - 1; j < n; i++, j++) {
        res += (hs(i, j) == ht(0, m - 1));
    }
    return res;
}
 
\end{lstlisting}
\end{multicols}
\end{document}
