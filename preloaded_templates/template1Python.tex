\documentclass{article}
\usepackage{listings}
\usepackage{xcolor}
\usepackage{multicol}
\usepackage[a4paper, total={6.5in, 10in}]{geometry}
%New colors defined below
\definecolor{codegreen}{rgb}{0,0,0}
\definecolor{codegray}{rgb}{0,0,0}
\definecolor{codepurple}{rgb}{0,0,0}
\definecolor{backcolour}{rgb}{1,1,1}

%Code listing style named "mystyle"
\lstdefinestyle{mystyle}{
  backgroundcolor=\color{backcolour},   commentstyle=\color{codegreen},
  keywordstyle=\color{codegreen},
  numberstyle=\tiny\color{codegray},
  stringstyle=\color{codepurple},
  basicstyle=\ttfamily\footnotesize,
  breakatwhitespace=false,         
  breaklines=true,                 
  captionpos=b,                    
  keepspaces=true,                 
  numbers=left,                    
  numbersep=5pt,                  
  showspaces=false,                
  showstringspaces=false,
  showtabs=false,                  
  tabsize=2
}

%"mystyle" code listing set
\lstset{style=mystyle}

\title{Cheatsheet}
\date{2023}
\author{ICPC}
\begin{document}

\maketitle
\section{Basic Python Template}
\begin{lstlisting}[basicstyle=\large\ttfamily,linewidth=\columnwidth,breaklines=true,language==Python]
#include <iostream>
'''input

'''
import sys
import math
import bisect
from sys import stdin,stdout
from math import gcd,floor,sqrt,log
from collections import defaultdict as dd
from bisect import bisect_left as bl,bisect_right as br

sys.setrecursionlimit(100000000)

inp    =lambda: int(input())
strng  =lambda: input().strip()
jn     =lambda x,l: x.join(map(str,l))
strl   =lambda: list(input().strip())
mul    =lambda: map(int,input().strip().split())
mulf   =lambda: map(float,input().strip().split())
seq    =lambda: list(map(int,input().strip().split()))

ceil   =lambda x: int(x) if(x==int(x)) else int(x)+1
ceildiv=lambda x,d: x//d if(x%d==0) else x//d+1

flush  =lambda: stdout.flush()
stdstr =lambda: stdin.readline()
stdint =lambda: int(stdin.readline())
stdpr  =lambda x: stdout.write(str(x))

mod=1000000007


#main code


\end{lstlisting}


\end{document}
